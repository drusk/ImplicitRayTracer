\documentclass[conference]{acmsiggraph}

\TOGonlineid{45678}
\TOGvolume{0}
\TOGnumber{0}
\TOGarticleDOI{1111111.2222222}
\TOGprojectURL{}
\TOGvideoURL{}
\TOGdataURL{}
\TOGcodeURL{}

\title{Ray Tracing Implicit Surfaces}

\author{David Rusk\\drusk@uvic.ca\\University of Victoria}
\pdfauthor{David Rusk}

\keywords{ray tracing, implicit surfaces}

\begin{document}

\maketitle

\begin{abstract}

A ray tracer for rendering implicit surfaces is developed.  It is
written in C++ and uses no external libraries.  The implementation of the
implicit surface ray tracer guarantees ray intersections will be found by 
placing upper bounds on the rates of change of the implicit function 
and its derivative following the procedure of Kalra and Barr \cite{KalraBarr1989}.

\end{abstract}

\keywordlist

%% Use this only if you're preparing a technical paper to be published in the 
%% ACM 'Transactions on Graphics' journal.

\TOGlinkslist

%% Required for all content. 

%\copyrightspace

\section{Introduction}

Implicit surfaces are a method for modelling objects using an equation
where the surface function is set to equal 0 as in equation \ref{eq:sphere} 
below.  There are two ways commonly used to
render these surfaces: polygonization and ray tracing.  Polygonization
involves breaking down the surface into polygons and rendering them, 
while ray tracing calculates the values of the pixels in the image
one-by-one.  This project investigates using the second technique, 
ray tracing, for rendering implicit surfaces.

The challenge when ray tracing implicit surfaces is reliably finding
ray intersections with the surface.  The techniques of Kalra and
Barr \cite{KalraBarr1989} were used to guarantee the success of this
process, as long as upper bounds on the net rate of change of the 
implicit function and its derivative are known.  

Kalra and Barr describe two steps to their method.  The first is a
space pruning algorithm which narrows down the search space for 
ray intersections.  The second is the ray intersection algorithm 
itself.  These algorithms will be explained in more detail in section 
\ref{sec:Algorithms}.

\section{Background}

As mentioned in the introduction, this project mainly implements the
techniques of Kalra and Barr \cite{KalraBarr1989}.  However, there have
been many other publications in this area, so a brief overview of some
of them will be given here.

Mitchell \cite{MitchellIntervals} proposed using interval arithmetic 
in order to find intersections reliably using two steps: root isolation
and then root refinement.  Many other authors worked on improving
performance of the interval method, such as Capriani et al. \cite{CaprianiIntervals} and
Florez et al. \cite{FlorezIntervals}.

Implicit surfaces have not been heavily utilized despite their advantages
(easy blending of surfaces, constructive solid geometry, collision detection),
mainly because they are expensive to compute.  Therefore, much research
has gone into improving computation speeds.  deGroot and Wyvill
\cite{deGrootWyvill} developed an algorithm called Rayskip which
exploits object space coherence and temporal coherence to reduce the
number of implicit function evaluations while ray tracing, thereby
improving rendering times.

Recently real time ray tracing of implicit surfaces has also become an active
area of research.  Singh and Narayanan \cite{RealTimeGPU} 
have developed a GPU-based real time implicit surface ray tracer.
Knoll et al. \cite{RealTimeIntervals} have developed 
a real time implicit surface ray tracer using SIMD optimized interval 
arithmetic and coherent ray traversal.  They claim it is able to
render any programmable implicit function based only on its definition,
in real time on the CPU without special hardware.  They also developed
a similar algorithm for use on the GPU \cite{RealTimeIntervals2}.

\section{Methodology}

This section will give some background information about ray tracing,
implicit surfaces, and LG surfaces, which will be useful for 
understanding the algorithms used in this project.

\subsection{Ray Tracing}

Rendering three-dimensional objects is a fundamental task in computer 
graphics.  The goal is to take a scene of three-dimensional objects 
and produce a two-dimensional image showing the objects as viewed
from a specific location.  In particular, the input is a set of 
objects and the output is an array of pixels.

There are two common ways to consider how each object contributes to each 
pixel: object-ordered rendering and image-ordered rendering.  In
object-ordered rendering iteration is over the objects, and for
each object all the influenced pixels are found and updated.  In
image-ordered rendering iteration is over the pixels, and for each
pixel the objects that affect it are found and used to calculate
that pixel's value.

Ray tracing is an image-ordered rendering technique.  For each pixel
it follows these steps:
\begin{enumerate}
	\item Compute the ray from the camera to the current pixel.
	\item Find the first object the ray intersects.
	\item Use the intersection point, the object's surface normal at that
	point, and any lighting, to calculate the pixel value.
\end{enumerate}

For more information about ray tracing, see Shirley et al. \cite{Shirley}.

\subsection{Implicit Surfaces}

Implicit surfaces are a method for modelling objects using an equation
where the surface function is set to equal 0.  For example, \ref{eq:sphere}
describes a sphere.

\begin{equation}
\label{eq:sphere}
f(x, y, z) = x^2 + y^2 + z^2 - r^2 = 0
\end{equation}

In this case the value of the function will be negative inside the 
sphere, positive outside, and 0 on the boundary.

In order to render the implicit surface, points in space are sampled
to see whether they are inside, outside, or on the border.  This process
is expensive to evaluate, but it has some advantages.  For more information,
see Shirley et al. \cite{Shirley}.

\subsection{LG Surfaces}
\label{sec:LGSurfaces}

Kalra and Barr \cite{KalraBarr1989} point out that it is not possible to
guarantee correct intersection of a ray with an arbitrary implicit surface
just using evaluation of the implicit function at sampled points in space.
This is because if the sampling is not fine grained enough, features that
are smaller than the distance between samples will be missed.

Therefore, extra information is needed to guarantee the intersections.
Kalra and Barr use Lipschitz constants to put bounds
on the net rate of change of the implicit function and its derivative along
the direction of the ray.

A Lipschitz constant is a constant that satisfies the inequality in 
equation \ref{eq:lipschitz}.

\begin{equation}
\label{eq:lipschitz}
|f(x_1) - f(x_2)| < LipschitzConstant * |x_1 - x_2|
\end{equation}

Let $f(x)$ be the implicit function, and let a ray be defined by its 
direction $\alpha$ and origin $\beta$ as in equation \ref{eq:ray}.

\begin{equation}
\label{eq:ray}
r(t) = \alpha * t + \beta
\end{equation}

Then the implicit function's value along the ray can be defined as in
equation \ref{eq:F}.

\begin{equation}
\label{eq:F}
F(t) = f(\alpha * t + \beta)
\end{equation}

The derivative of the original implicit function along the ray direction is 
then as shown in equation \ref{eq:g}.

\begin{equation}
\label{eq:g}
g(t) = dF/dt
\end{equation}

Then let $L$ be the Lipschitz constant for $f(x)$, and let $G$ be the Lipschitz
constant for $g(t)$.  

\begin{equation}
\label{eq:L}
|f(x_1) - f(x_2)| < L * |x_1 - x_2|
\end{equation}

\begin{equation}
\label{eq:G}
|g(t_1) - g(t_2)| < G * |t_1 - t_2|
\end{equation}

Using this background information, an LG surface can
then be defined as an implicit function where $L$ and $G$ exist and can be
computed.  Kalra and Barr's algorithm guarantees correct ray intersections
for LG surfaces.

\section{User Interface}

The current implementation does not have an interactive graphical user 
interface because ray tracers generally are not used in real time,
though that is an active area of research (see references in the Background
section).  Instead the application is run from the command line with constant 
parameters such as image dimensions defined in a special header file.
The scene to be rendered is also currently specified directly in the 
code.  This is less than ideal because it requires recompiling the project to 
change the scene.  Some domain specific language should be developed
for specifying the scene, which the application can parse and
use to build the scene.  It should also be possible to pass arguments
on the command line to change the different parameters like
image dimensions.

\section{Data Structures}

This section describes some of the data structures and interfaces 
used to implement the implicit surface ray tracer.  Only two of the 
more fundamental and interesting ones are presented here.

\subsection{Implicit Surface Interface}

The system has an interface class for implicit surfaces.  To add a new
type of implicit surface to the system, simply create a subclass
which implements the interface's methods.  The C++ code for the interface
is shown below:

\begin{verbatim}
class ImplicitSurface
{
public:
    /**
     * Returns a bounding box around this 
     * surface.  It does not have to be the 
     * minimal bounding box.
     */
    virtual Box BoundingBox() = 0;
    
    /**
     * Returns the value of the implicit 
     * function that defines this surface
     * at the specified point in space.
     */
    virtual double ImplicitFunction(
            Vector3D point) = 0;
    /**
     * Returns the gradient of the 
     * implicit function that defines this 
     * surface at the specified point in 
     * space.
     */
    virtual Vector3D Gradient(
            Vector3D point) = 0;
    
    /**
     * Returns the gradient of the 
     * implicit function in the direction
     * of the ray, at the point t along 
     * the ray.
     */
    virtual double DirectionalGradient(
            Ray ray, double t) = 0;
        
    /**
     * Returns the Lipschitz constant for 
     * this surface in the region defined 
     * by the minPoint and maxPoint.
     */
    virtual double LipschitzConstant(
            Vector3D minPoint, 
            Vector3D maxPoint) = 0;

    /**
     * Returns the Gradient Lipschitz 
     * constant for this surface along the
     * specified ray, between ray 
     * distances t1 and t2.
     */
    virtual double GradLipschitzConstant(
            Ray ray, 
            double t1, double t2) = 0;
};
\end{verbatim}

\subsection{Octree}
\label{sec:Octree}

The octree is a tree data structure where each node has up to eight 
children.  It is used in the space pruning algorithm described in
section \ref{sec:SpacePruning}.  Each node in the tree represents a volume
in space, and child nodes are subsections of this volume.  To create
child nodes, the current node's volume is split into eight equal pieces.
A similar data structure which can be visualized in two dimensions is
the quadtree which has four children per node.  Figure \ref{fig:quadtree}
visualizes the recursive nature of the space subdivision, and how that is translated
into a tree structure. 
The octree works the same way, but in three-dimensions by splitting in
the z-axis as well.

\begin{figure}[ht]
  \centering
  \includegraphics[width=1.5in]{figures/quadtree.png}
  \caption{Example of a quad tree's spacial representation and tree representation.}
  \label{fig:quadtree}
\end{figure}

The octree structure aids in finding the first box a ray intersects (needed 
for the ray intersection algorithm given in section \ref{sec:RayIntersection}.
This is because if it is discovered that the box associated with a node does not
intersect with the ray, then all of its children can be ignored as they
definitely will not intersect either.

\section{Algorithms}
\label{sec:Algorithms}

This section describes the ray intersection method developed by
Kalra and Barr \cite{KalraBarr1989}.  They break their method down into
two algorithms: space pruning and ray intersection.

\subsection{Space Pruning}
\label{sec:SpacePruning}

The space pruning algorithm is run first, and its primary purpose is to
improve efficiency.  The basic idea is that the algorithm will prune 
away parts of space that definitely do not contain any piece of the LG
surface, and thereby reduce the space that needs to be searched for 
intersections.  This algorithm is not actually needed for correctness,
the ray intersection is still guaranteed to be calculated correctly
even if no pruning is done.

The algorithm proceeds as follows:
\begin{enumerate}
	\item Start with a bounding box around the surface.  It does not
	have to be a tight bounding box, it just needs to contain the
	entire surface.
	\item Recursively subdivide the box to some level n.
	\item Keep sub-boxes that contain part of the surface, as shown
	in Figure \ref{fig:surface_boxes}.
\end{enumerate}

\begin{figure}[ht]
  \centering
  \includegraphics[width=1.5in]{figures/surface_boxes.png}
  \caption{Sub-boxes containing part of the implicit surface.}
  \label{fig:surface_boxes}
\end{figure}

This overview of the algorithm leaves out the details of how it is 
determined whether a sub-box contains part of the surface.  This will
be explained next.

A box is said to straddle the surface if it has at least one corner
inside the surface and one outside.  Figure \ref{fig:straddling_boxes}
shows a few examples of boxes where green ones straddle the surface
and red ones do not.

\begin{figure}[ht]
  \centering
  \includegraphics[width=1.5in]{figures/straddling_boxes.png}
  \caption{Straddling boxes are shown in green, non-straddling are shown in red.}
  \label{fig:straddling_boxes}
\end{figure}

If a box straddles the surface, it definitely contains a part of the 
surface and is kept.  However, even if it does not straddle the surface,
it still might contain a portion of the surface, such as the red box at
the function peak in Figure ~\ref{fig:straddling_boxes}.

If a box does not straddle the surface, some more calculations have to
be performed in order to decide what to do with it.  First let us 
define $x_{0}$ to be the center of the box, and $d$ to be the half-length
of the principal diagonal of the box, as shown in Figure \ref{fig:box}.

\begin{figure}[ht]
  \centering
  \includegraphics[width=1.5in]{figures/box.png}
  \caption{A box showing the definition of $x_0$ and $d$.}
  \label{fig:box}
\end{figure}

Let $L$ be defined as in equation \ref{eq:L}.  Essentially it is
the maximum rate of change of the implicit function $f(x)$.  

\begin{equation}
\label{eq:LConstraint}
|f(x_0)| > L * d
\end{equation}

If equation \ref{eq:LConstraint} is satisfied, then $f(x)$ must stay 
the same sign and never assume a value of 0.  
This means there is no zero-crossing (the implicit function
evaluates to 0 on its surface).  Therefore the box definitely does not
contain part of the surface and can be pruned.  

If equation \ref{eq:LConstraint} is not satisfied, then the box is 
subdivided into eight, and each sub-box is checked.

The end result of the space pruning is an octree data structure 
(see section \ref{sec:Octree}) with the boxes completely enclosing the
implicit surface.

\subsection{Ray Intersection}
\label{sec:RayIntersection}

The second algorithm from Kalra and Barr is the ray intersection algorithm 
itself.  From the space pruning algorithm, a set of boxes that completely 
encloses the surface have already been found.  Therefore all intersections
of rays and the surface will occur in the volume of this set of boxes.

The basic idea is to find the closest box to the ray origin (along the 
direction of the ray), then compute the intersection in this box.  If 
there is no intersection, the next nearest box is checked.  Finding 
the closest box is aided by keeping the boxes in an octree as
discussed in section \ref{sec:Octree}.

Let $t_1$ and $t_2$ be two points along the ray (see definition of ray
in equation \ref{eq:ray}).  Then let $t_m = (t_1 + t_2) / 2$ and 
$d = (t_2 - t_1) / 2$.  
$F$ is as defined in equation \ref{eq:F} and $g(t)$ is defined as in 
equation \ref{eq:g}.  Recall that $G$ is the Lipschitz constant for $g(t)$,
as defined in equation \ref{eq:G}.

\begin{equation}
\label{eq:GConstraint}
|g(t_m)| > G * d
\end{equation}

If $F(t_1)$ and $F(t_2)$ have opposite signs and the constraint in 
equation \ref{eq:GConstraint} is met, then there is exactly one 
intersection between $t_1$ and $t_2$ and it can be calculated with
a numerical method such as Newton's method.

If $F(t_1)$ and $F(t_2)$ have the same sign and the constraint in
equation \ref{eq:GConstraint} is met, then there is no intersection
between $t_1$ and $t_2$.  Therefore the next box is considered,
and if there are no more boxes, it can be concluded that the ray
does not hit a surface.

There is one more case to consider: when the constraint in equation
\ref{eq:GConstraint} is not met.  Then the interval is subdivided into
intervals [$t_1$, $t_m$] and [$t_m$, $t_2$], and they are tested.
This condition can occur when there is more than one intersection
in the box.

Subdividing stops when the interval distance $d$ falls below the
tolerance of the numerical method used to compute the intersection.

\section{Implementation}

There are many open source ray tracers available on the internet.  However, 
for pedagogical reasons it was decided to implement a simple ray tracer that
did not depend on external libraries.  
This also made it easier to modify and adapt to support implicit
surfaces because many of the open source ray tracers are part of larger 
packages or have a lot of additional functionality which makes them 
complicated to use.

The implementation was done in C++.  Output 
images were written to PPM files because the PPM format is 
very simple and can be implemented trivially without external libraries.

The project source code is available online at 
https://github.com/drusk/ImplicitRayTracer

\section{Results}

The system developed can currently render spheres with adjustable locations,
radii, colours, reflectivity and transparency.  Figure \ref{fig:spheres} shows a
simple example.  The ground is made using a sphere with a very large radius.
The light source (not in the field of view) is also a sphere.  Except for the
ground sphere, all of the spheres are reflective.  Only the green sphere has some
transparency.

\begin{figure}[ht]
  \centering
  \includegraphics[width=1.5in]{figures/spheres.png}
  \caption{Example of ray traced spheres.}
  \label{fig:spheres}
\end{figure}

\section{Conclusion}

An implicit surface ray tracer was developed from scratch with no external
dependencies.  It uses the techniques of Kalra and Barr \cite{KalraBarr1989}
to find guaranteed ray intersections with the implicit surfaces.  This is done
by using two values: "L" which is an upper limit on the rate of change of the implicit
function, and "G" which is an upper limit on the rate of change of the gradient.  This is significant because it opens up ray tracing as an 
alternative to polygonization when rendering implicit surfaces.

\section{Future Work}

A major area of future work is calculating the Lipschitz constants L and G
for other useful implicit functions.  Kalra and Barr \cite{KalraBarr1989} 
mention a forthcoming report with additional Lipschitz constants, but to
our knowledge no report was ever published.  This will allow new types of
surfaces to be rendered, and greatly increase the utility of implicit 
surfaces for modelling when using ray tracing.

Future development of the application could add support for blending surfaces
which are close to each other.  A simple two-dimensional example of this concept
is shown in Figure \ref{fig:blend2d}.

\begin{figure}[ht]
  \centering
  \includegraphics[width=1.5in]{figures/blend2d.png}
  \caption{Two-dimensional blend of two circles.}
  \label{fig:blend2d}
\end{figure}

Adding constructive solid geometry (CSG) operations would also be useful
for modelling more advanced structures.  CSG combines primitives like 
spheres and cylinders by using the boolean set operations union,
intersection and difference.

\bibliographystyle{acmsiggraph}
\bibliography{ImplicitSurfacesRayTracer}
\end{document}

